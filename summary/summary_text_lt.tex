Tiriamojo darbo objektas yra adaptyv\={u}s mokymo metodai, kurie remiasi kryptingu mokymo imties formavimu. 
Patobulintos \v{z}inomos mokymo strategijos esant staigiems, palaipsniams ir pasikartojantiems poky\v{c}iams. 
Sukurti ir eksperimenti\v{s}kai aprobuoti keturi adaptyvaus mokymo imties formavimo algoritmai, kurie leid\v{z}ia pagerinti klasifikavimo bei prognozavimo tikslum\k{a} besikei\v{c}ian\v{c}iose aplinkose, esant atitinkamai kiekvienam i\v{s} trij\k{u} poky\v{c}i\k{u} tip\k{u}. 
Naudojant generuotus bei realius duomenis, eksperimenti\v{s}kai parodytas klasifikavimo bei prognozavimo tikslumo pager\.ejimas, lyginant su vis\k{u} istorini\k{u} duomen\k{u} naudojimu mokymui, bei \v{z}inomais \v{s}ioje srityje naudojamais adaptyviais mokymo algoritmais. 
Sukurta metodika pritaikyta pramoninio katilo atvejui, jungian\v{c}iam kelis aplinkos poky\v{c}i\k{u} tipus. 

...